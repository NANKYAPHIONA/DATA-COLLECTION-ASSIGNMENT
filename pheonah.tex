\documentclass[14pt]{article}

\usepackage{graphicx}
\begin{document}
 \begin{huge}REGISTRATION NUMBER: 15/U/21729\end{huge}
 \begin{huge}STUDENT NUMBER: 215023051\end{huge}
  \begin{huge}INSTRUCTOR: ERNEST MWEBASE\end{huge}
 \begin{huge}BACHELOR OF COMPUTER SCIENCE\end{huge}
\title{THE CHALLENGES THAT STUDENTS INCUR WHEN GETTING APPLICATION FORMS FROM SENATE BUILDING}

 

\author{BY NANKYA PHIONA }
\maketitle 

\date{\today}
 
\tableofcontents

\section{ACKNOLEDGEMENT}
I would like to give my sincere appreciation to the students who took their time to give the relevant information about my project and mostly I take my great honor to thank my lecturer Mr. Ernest Mwebaze for this project because it has taught me to a lot, most especially to be a hardworking person.

 \subsection{Terms of references}
To put down the challenges that we students face when we come to pick application forms from Senate building at Makerere University.

 \subsection{Procedure}
The procedures I went through during the analysis of my project and come up with this report are as follows;\par
1.	Downloading the ODK collect on my android device\par
2.	Went to my google account and searched “google cloud platform” using the google search engine.  \par
3.       Open console link option in the right corner of the page and created a new project\par
4.	Downloaded the ODK aggregate on my Laptop which I used to fill in some simple thing s to come up with the aggregate server which was successfully done.\par
5.	Opened “build.opendatakit.org” signed in and created the xml form which I downloaded and transferred it to the ODK collect folder in the phone using the a USB device\par
6.	Opened the fill blank space in the menu of ODK collect that led to my xml form.\par
7.	Started interviewing students according to my project and questions I created in my xml form.


 
 

\section{INTRODUCTION}
The main purpose of this report is to address the challenges that students face when they come to pick the forms from Senate building at Makerere University.\par
   By examining the challenges I faced when getting the application form in May 2015, this report is to describe the same challenges that my fellow students face to get application forms manually from Senate building though having Online registration site which is also challenging to many students, the report also talks about what the management should do to reduce the queues at that place
\subsection{Finding}
Students applying to make Makerere University incur problems that any person cannot imagine, to my experience, observation and analysis I went through, I came up with the following findings;\par
1.	Short time given for applying the relevant courses. Students applying are given a few weeks for example two weeks which is not enough for thousands of students who came from different parts of the country thus a big challenge to students\par
2.	Date to apply. The published time to apply or picking the forms is announced late make students unprepared to get the forms from the university, which creates a lot of congestion at the site about my project.\par
3.	University Management working registration issues; The academic management at level3 is so challenging in that it doesn’t time to students, they sometime find some officers closed and others claim it’s a weekend without considering the time students spend coming for those forms and the transport incurred.\par
4.	Online Registration. Though online registration was launched but its more challenging to the new users because its not user friendly, having a lot of procedures and that’s why many students run for the manual registration.\par
Table showing number of students who used manual registration to Online

\begin{table}[h]
\centering
\begin{tabular}{c c}
\hline
Types of registration&	Number of students\\[0.5ex]
\hline
Manual & 10\\
\hline
Online &  2\\
\hline
Total number of students interviewed&	12\\
\hline
\end{tabular}
\end{table}

\subsection{Solutions to the above problems}
To my research carried out, came up with the following solution to the challenges that students face;\par
1	The management of Makerere University doing registration issues should at least be given to students a long period of time to register for example like a month and that will reduce on the overcrowding of students in the building.\par
2	The management should always be in position to announce the application time early such that people should be informed and proceed the registration process in the month of April.\par
3	The management should at least be so fair and know the challenges that students face when they spend their money to come at the University for those Forms and this will reduce overcrowding.\par
4	The Online registration process be made user friendly, and the Systems Admini9stration should know what type of language that can fit the users

\section{Methodoligies}

   \subsection{Materials and Methods}
Equipment used in my project;\par
An android phone\par
Laptop\par
Internet cable to connect to internet
 \subsubsection{Methods;}
1.Interviewing using the form I created in my ODK build which acted as my questionnaire\par
2.Sampling; I got different students offering different courses


\subsection{Results}
The results about my research project were so massive in that they are tired of overcrowding at Senate Building, but the managemennt will try to go through these issues faced by the students.

\section{Functionality and Screenshots}
My data was collected electronically using the android phone that connected to the aggregate server and i came up with the following screenshots which is an implication that shows that my project was successful



\begin{figure}[h!]
\includegraphics[width=100mm,scale=0.5]{1.jpg}
\caption{app engine.}
\label{figure1}
\end{figure}

\begin{figure}[h!]
\includegraphics[width=100mm,scale=0.5]{3.jpg}
\caption{odk form.}
\label{figure2}
\end{figure}


\subsection{Conclusion}
This report has identified the problems that students incur when getting application forms from Senate building, it also represents what the management of our mighty Makerere University dealing with registration should do to overcome the challenges mentioned.

\subsection{Recommendations}
1.	Balintuma Isaac, student at Makerere University doing COMPUTER SCIENCE

\begin{thebibliography}{10}
\bibitem{latexGuide} The Students
  \texttt{The community around Makerere University}

\bibitem{latexGuide} New Students
  \texttt{Found at Senate Building}

\end{thebibliography}

\end{document}